\section{Feasibility Study \& Risk Analysis}
When considering the execution of a project, its feasibility and possible risks are tantamount to the design and engineering itself. For this reason, this section will cover, first, a feasibility study about the scope and context of the project (2.1), covering both technical and non-technical feasibility, followed by a risk analysis (2.2).

\label{sec:feasability&risk}
\subsection{Feasibility Study}
%%By now you should have a pretty good overview of the problem and what the client wants / needs. Consider the following questions and write a section on the feasibility of the project:
%•	Can this be realized given the time constraints and the constraints on your resources? Is the project too small? Too large? Too complex?
%•	Are technologies/frameworks/data available to realize the project? 
%•	What changes can be proposed to the client in case the current project is infeasible to realize?

\indent \indent When considering technical feasibility, the scope and context of our project are vital. Creating a full-stack web application with a translation tool in 10 weeks seems to be feasible, albeit somewhat challenging. We are very comfortable working with full-stack applications since we have had relevant courses (Object Oriented Programming Project, Software Engineering Methods, and Web \& Database Technology) that extensively cover the development of full-stack applications. The creation of a ML model that accurately translates a LRL seems to be the part of the project that takes the most time and effort. While we have done courses (namely Machine Learning, Data Mining, and Computational Intelligence) whose curriculum includes a lot of the theory behind the basics of the implementation of these technologies, there are still various technologies, services, and theories that we haven’t covered in our courses so far (for example, Transformer based models and \gls{LSTMs}). Research into these further topics will be key, however, we have access to AWS services, which drastically cuts down on the amount of time needed on creating a viable ML model, and all the specifics with a full-stack web application, while still allowing for scalability and continuous deployment. Without going into too much detail on the specific services we intend to use, information on this will be provided in \href{#sec:projectapproach}{Project Approach} and \href{#sec:developmentmethodology}{Development Methodology} sections.  

\subsubsection*{Technical Feasibility: SWOT Analysis}
% Can we physically make an ML model for low-resource translation

\indent \indent The following analysis will be conducted regarding the strategic planning of the project on its strengths, weaknesses, opportunities, and threats to evaluate the planning and design of the project's solution: 

\begin{itemize}
  \item Strengths: 
  The most significant strength of this project is the availability of AWS technologies provided by AWS. The following services described will be used to build the full-stack web application: AWS SageMaker and AWS Amplify among other services which may be of use in our application. The use of these services will optimize the cost as well as the time spent on certain tasks. \cite{ChatGPT}. 
  \item Weaknesses: The dependency on the AWS services could limit the design choices for the web application due to the limited customization options. Due to working with a low-resource language (Turkish), there could be limited training data available to train the ML algorithms which may affect the reliability and quality of the translation. Due to limited resources, the actual meaning of sentences from Amazon blog posts may not be captured accurately, leading to an underperforming and under-providing translation \cite{ChatGPT}. 
  \item Opportunities: The translation of the blog posts can have a significant outcome in terms of expanding AWS's market since the use of AWS services may increase due to the increase in the availability of technical information on AWS to Turkish users. AWS's business can perform better in the regions with Native Turkish Speakers. After the completion of the web application, the process of translating other low-resource languages will be made easier, hence allowing AWS to expand its business with new opportunities. Lastly, the use of AWS services throughout this project will enhance the developers' understanding of ML and cloud computing while allowing them to improve their technical skills and benefit from these services in the implementation of their web application \cite{ChatGPT}. 
  \item Threats: Other translation services such as Google Translate could be used instead of our translation service since customers could find them more reliable due to their long-term existence and use. Due to the low-resource nature of the language, the training data may not be enough to build an accurate model resulting in translation mistakes which could reduce the reliability and use of the application. The level of experience of the developers of this project could concern the customers on the quality of the translation service, leading to fewer customers using it. The use of AWS and the personal accounts could potentially pose a problem in the case of misuse which may result in financial damage to AWS \cite{ChatGPT}. 

  There are various different aspects of this project which need to be considered while implementing the web application, especially from a security, reliability, and maintenance approach. 
\end{itemize}

\subsubsection*{Non-Technical Feasibility}
Besides the analysis regarding technical feasibility, there are also certain things to consider on the non-technical side of this project. Technical support is a significant part since there will be AWS experts with high availability to consult on the progress of the project, especially for guidance with the services. The impact of this application on society should also be considered since this project will highly affect users of AWS, specifically Native Turkish Speakers, that couldn't have high access to information on AWS before. The understanding of low-resource languages in terms of their definition as well as what hinders them from becoming high-resource should be well thought out to support the development process of the ML model. Turkish as a language should be analyzed for its difficulties that could possibly pose certain risks to the accuracy of the translation service. 


% Learning how to 
% use AWS services
% How ML models work
% make the front-end, back-end, and gateway

% Discuss more about the ML stuff and how Sagemaker can help

%tackle each individual part of the stack

\subsection{Risk Analysis}
%%Risk analysis
%This can be interpreted broadly; consider all potential risks to the successful completion of your project. For example: Is the client continuously available during the project? Do you need specific hardware or data to complete the project (which you don’t have yet)? Are there potential legal issues (ownership of data, privacy constraints)? Do you as a team have the required knowledge (and help) to complete the project, or could you get stuck? 
%For a successful project all risks should be relatively low or be mitigated in some way!
\indent\indent It is paramount to consider all potential factors that may hinder, or outright prevent, the final delivery of the project. For this reason, we have considered a set of possible risks and their danger of interfering with our ability to meet the client's needs. Some of these risks are simply outside of our control while others are reasonably in our hands, and can therefore be mitigated. 


%Used at the time of writing, change time and info by the time we have finalized the report
\indent In the scope of this project, we have a hard dependency on using the client's AWS services. This is an integral part of our project and failure to do so is considered a failure to complete the project, however, this dependency poses a potentially low risk as AWS is considered to be incredibly reliable \cite{leopold2015}. Additionally, we have a risk concerning the quality of the data for creating our translation tool (through SageMaker). If the client is unable to provide us with sufficient quality data to train the model, this core requirement is unfeasible. This is considered a \textit{medium-level risk} because we are translating \textbf{to} an \acrshort{LRL} and the AWS Turkish Blog Data-set in it's entirety may not be enough to train a full-blown translation model.

\indent The team needs to develop a scalable, well-tested, full-stack application utilizing unfamiliar AWS services within 10 weeks while only being able to meet the client in-person weekly. Inadequately planning the project, time management, collaboration, or motivation may lead to a poor final product or failure altogether. Being evaluated at a \textit{low-medium level risk} means the team will not only need to use its resources wisely but also ensure all tasks are properly planned and completed successfully in a timely manner.



