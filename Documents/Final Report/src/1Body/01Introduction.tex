\section{Introduction}
\indent \indent An estimated 80\% of the internet’s content is available in only 10 of the identified 7000 languages \cite{moseley2010}\cite{eberhard2021}. This means that the world's population has limited to no access to digitally translated content in their native tongues. Low-Resource Languages (LRLs) are often marginalized in the digital space and, as a result, the native-speakers of these languages have limited access to tools and opportunities on the Internet. For the Software Project, we are working for Amazon Web Services (AWS). AWS is a subsidiary of Amazon.com, Inc. that offers cloud services such as computing power, storage for databases, and other technologies that provide reliable, scalable, and cost-optimizable services. As part of their infrastructure, the client provides valuable AWS Blogs showing its users valuable information related to their cloud-computing infrastructure. AWS wants to make this information available to all and has therefore asked us to design an application which takes an AWS blog and is able to translate it to a target LRL. This report aims to present the research conducted and implementation process of designing and developing a full-stack application capable of translating AWS Blog Posts into an LRL using either Amazon Translate or a neural translator designed by the group. The purpose of this research was not just to devise a solution, but to make AWS blog content more accessible and inclusive, thereby bridging the linguistic divide.

\indent The primary aim of our research is to answer the question: Can we create a full-stack application that translates AWS Blog Posts into a low-resource-language effectively using Amazon Translate or a custom in-house model built and trained on Amazon SageMaker? This report will discuss the process of implementing all parts of the application as well as training our model to translate English AWS blog posts into Turkish (a LRL) in the specific AWS tech domain. 

\indent This report has been structured to provide a comprehensive understanding of our research and its outcomes. It begins in Chapters 1 and 2 with a detailed overview of the theoretical underpinnings of machine translation and the challenges associated with translating LRLs. Chapters 3 and 4 explores the research identifying key resources and limiting conditions such as software and data limitations. This is followed in Chapters 5 and 6 by a description and implementation of our methods, including all sub-level concepts and explanation of the different AWS services used in development of the application. Subsequently, in Chapter 7 and 8 discuss the testing and evaluation of our final product, effectively comparing translations generated by Amazon Translate and those by our own model. The final sections cover the implications of our research, potential future work, and our conclusions. We then explore the ethical and societal implications of our work in Chapter 9 before concluding our findings in Chapter 10.
