%%%%3 Requirements 
\section{Requirements}
\label{sec:Requirements}

\indent \indent Requirements are an essential part of project planning. It is a way to ensure the joint comprehension of the client and developers of the expected functionality of the deliverable. We have created a list of the requirements we identified using the \gls{russia} model. Given that we worked closely with the client, we believed this form of collaborative (between client and stakeholders) requirement elicitation through the MoSCoW method would most benefit this project \cite{moscow}. Following the MoSCoW model, we divided our requirements into functional and non-functional categories.
%Create a list of all requirements and order them using the %MoSCoW model. Ensure that the list of “Must haves” you create is a set of requirements you expect to finish by weeks 7-8 of the project under normal circumstances. 
%Also list any relevant non-functional requirements.

\subsection{Functional Requirements:}

\subsubsection*{Must Have}

\begin{itemize}
\item Translation: The application must have the following translation requirements
\begin{itemize}
\item The application must translate AWS blog posts into a low-resource language (Turkish).
 \item If the blog post contains code, translation must be avoided and code should remain as is.
\item Title of the blog post must be translated.
\item User must be able to pick between 2 translation modes.
\begin{enumerate}
\item Amazon Translate
\item \gls{ntm} built off Amazon SageMaker (This is technically a must have, however, due to the nature of training \gls{nlp} machine learning models, it will not be ready by the MVP deadline)
\end{enumerate}
\end{itemize}
\item Input: The user must be able to enter a URL.
\item Output: The user must be able to view the translated document.
\end{itemize}

\noindent \subsubsection*{Should Have}

\begin{itemize}
\item Input: The following functionality should be available when inputting a request.
\begin{itemize}
\item Technical jargon should not be translated.
\item Application should allow the user to search/toggle between output language(s).
\item Application should allow the user to toggle between translation techniques.
\item Application should only allow AWS blog links to be pasted.
\end{itemize}
\item Output: The user should get the original post on \acrshort{lhs} and translated post on \acrshort{rhs}.
\begin{itemize}
\item Posts should be scrollable and have elements aligned on both sides.
\end{itemize}
\item Interaction: The application should allow for interactive functionality on the output:
\begin{itemize}
\item User should be able to highlight parts of the input (and reflect on the output).
\end{itemize}
\end{itemize}

\noindent \subsubsection*{Could Have}

\begin{itemize}
\item Input: The following functionalities could be added when inputting requests.
\begin{itemize}
\item Application could allow for the user to 'search' for blogs by title if there is metadata on them.
\item Application could automatically infer the blog's language (Assuming translating from more languages than just English).
\end{itemize}
\item Output: The functionalities of the website (hyperlinks and share) could be provided in the translation.
\item Interaction: The following interactive functions could be implemented.
\begin{itemize}
\item User could export the blog with its translation and print it. 
\item User could be able to switch between translation modes seamlessly.
\begin{enumerate}
\item If currently viewing Amazon Translate translation, the user could seamlessly switch to NLP translation (and vice versa) with a button, etc.
\item Same positions of elements could remain on both sides.
\end{enumerate}
\item Application could have a way for users to give star ratings on how well the blogpost was translated
\item Highlighting/editing sections could be seamless on both sides (no need to re-translate).
\end{itemize}
\item Authentication: The application could have a form of authentication with the following.
\begin{itemize}
\item Application could allow users to register (using social media SSO's).
\item Application could allow users to identify themselves to view/save previous notes/translations/edits.
\end{itemize}
\item Translation: The blog post's author and information about them could be translated. 
\end{itemize}

\noindent \subsubsection*{Won't Have}
\begin{itemize}
\item Translation: 
\begin{itemize}
    \item Translation won't have support for multiple languages and will only be one way (English to Turkish).
    \item The comments of blog users won't be translated.
\end{itemize}
\item Interaction: The user won't be able to edit the blogs or the translated version of the blogs.
\end{itemize}


\subsection{Non-Functional Requirements:}

\subsubsection*{Must have}
\begin{itemize}
\item Performance: The application must have a fast response time on all functionalities.
\item Reliability: Application must have limited to no downtime or disruptions.
\item Compatibility: The application must be accessible across popular browsers/operating systems on a desktop.
\item Usability: The application must be easy to use. Users intuitively know how to do all functionalities.
\item Translation: The translations provided by the \acrshort{NLP} are accurate and natural to target \acrshort{LRL} (Turkish).

\item Technology: AWS services will be used, in particular, Amazon Translate for the translation and Amazon SageMaker to fine-tune the ML model. An AWS service of choice must also be used to deploy and host the application. 
\item Data Privacy: User data must be well protected, adhering to regulations. (\href{https://gdpr-info.eu/}{GDPR}).
\item Re-usability: The project must be built in a way that ensures future developers can easily continue development and feature integration.
\item Scalability: The Application must be easily scalable to add new functionalities.
\begin{itemize}
\item Application must be able to handle multiple users at a time and should scale to handle upto 100 users.
\item Software must allow easy integration of translation to other \acrshort{LRL}s.
\end{itemize}

\end{itemize}

\subsubsection*{Should Have}
\begin{itemize}
\item Documentation: Codebase should be well documented using a coherent format throughout.
\item Visually: The application should be aesthetically pleasing and similar to other AWS websites.
\item Accessibility: The application should be accessible for people with disabilities following \href{https://www.w3.org/TR/WCAG21/}{WCAG2.1}
\item Testing: The application should cover the following testing.
\begin{itemize}
\item unit testing
\item integration testing
\item functional testing
\item acceptance testing
\end{itemize}
\item Technology: The following technology should be used.
\begin{itemize}
\item An AWS service should be used to store data (like DynamoDB). This database should be for blogposts, though the concrete use may vary.
\item Java as the main language for the project.
\item Python as the secondary language primarily used for NLP and any other possible AI/machine learning functionalities.
\item A CI/CD pipeline should be used in the application and in the infrastructure of the code.
\end{itemize}
\item Security: The application should defend against common cyber attacks.
\begin{itemize}
    \item DDOS attack security
    \item Injections (if we are using a database)
\end{itemize}
\item Scalability: Software should allow easy integration of new translation methods. (Beyond Amazon Translate and NLP).
\end{itemize}

\noindent \subsubsection*{Could Have}

\begin{itemize}
\item Translation: The translations provided by the NLP are capable of translating technical jargon in a way understandable by LRL
\item Monitoring: The application could have a form of monitoring aspects like performance and reliability. 
\begin{itemize}
\item Monitoring performance and reliability could include: checking the compiler, execution, invocation, duration, error comments, success rate, and connecting to \gls{Amazon CloudWatch}. 
\item Alerts could notify developers of any failures in the system.
\end{itemize}
\item Authentication: The application could have the option for users to pay for additional services which include logging in and saving/downloading the translations. 
\end{itemize}

\noindent \subsubsection*{Won't Have}

\begin{itemize}
\item Accessibility: The application will have the following accessibility restrictions. 
\begin{itemize}
\item The application won't have a mobile version
\item The application won't be downloadable (web application only).
\end{itemize}
\item Usability: The application won't have instructions that guide users through the process.
\end{itemize}
